\documentclass{article}    % Specifies the document style.
\renewcommand{\baselinestretch}{1}      % Spacing
\setlength{\oddsidemargin}{0in} \setlength{\evensidemargin}{0in} \setlength{\textwidth}{6.5in}
\setlength{\topmargin}{-.2in} \setlength{\headsep}{0in} \setlength{\textheight}{9.5in}
\pagestyle{empty}
%
\usepackage{color}
\usepackage{amsmath,amsthm}
\usepackage{graphicx}
\usepackage{enumitem}
\usepackage{url}

\newcommand{\answer}[1]{{\color{blue}#1}}
\newcommand{\grade}[1]{\textcolor{red}{#1}}
\renewcommand{\grade}[1]{ }	%to remove grading scheme, just make this link active
%\renewcommand{\answer}[1]{ } %%to remove answer , just make this link active

\begin{document}           % End of preamble and beginning of text.
\begin{center}

\textbf{\large{ENGRD 2700: Basic Engineering Probability and
    Statistics\medskip \\ Fall 2019}}\bigskip

\vskip 0.25em

\textbf{\large{Homework 1}}
\end{center}

\vskip 1em

\noindent Due Friday, September 13 by 11:59 pm. Submit to Gradescope by clicking the name of the assignment. See \url{https://people.orie.cornell.edu/yudong.chen/engrd2700_2019fa.html#homework} for detailed submission instructions. 

\vskip 1em

\noindent When completing this assignment (and all subsequent ones), keep in mind the following:
\begin{itemize}	
	\itemsep0em
	\item You must complete the homework individually and independently.
	%\item Clearly write your name, NetID, and section number/time on the first page of your submission.
    \item Provide evidence for each of your answers. If a calculation involves only very minor computation then explain the computation you performed and give the results. If a calculation involves more complicated steps on many many records then hand in the calculations and formulas for the first few records only.
	\item Write clearly and legibly. You are encouraged to \emph{type} your work although you do not have to. We may deduct points if your answers are difficult to read or disorganized.
	\item For questions that you answer using R, attach any code that you write, along with the relevant plots. You may use other software, but the same condition applies.
	\item Submit your homework a single pdf file on Gradescope.
\end{itemize}

\vskip 1em
\begin{enumerate}
\item The file \texttt{Quartet.csv} contains four datasets of $x$ and
  $y$ values, side by side.

	  \begin{enumerate}
	  	\item Compute the sample mean, sample median, and
                  sample standard deviation for each column of the dataset.
	    \item Based solely upon the summary statistics you computed in part (a), how do the four datasets compare? 
	    \item Construct scatterplots for each of the four
              datasets. (Hint: In R, you can use the command \texttt{par(mfrow=c(2,2))} to combine multiple plots into a single 2-by-2 graph in R. If you do, this command should precede any code that you use to generate plots.)
	    \item Based solely upon the plots you generated in part (c), how do the four datasets compare? 
	    \item What's the moral of the story? (That is, what does this example suggest about what should be done when analyzing data?)
	  \end{enumerate}

\vskip 1em

\item Answer the questions below about the dataset \texttt{CountyData.csv} from the
U.S. Census Bureau, performing any data analysis you deem
appropriate. The dataset consists of 3143 observations on 53 variables, which are
described in the file \texttt{CountyData\_Info.pdf}.

\begin{enumerate}
\item Provide a histogram of the per-county percentage of residents who speak a foreign language at home during 2006-2010.

\item What was the median per-county amount of federal spending in 2009?

\item Create a scatter plot of the percentage of residents below the
  poverty level ($y$-axis) versus the percentage of the population
  with a bachelors degree. Comment on what you see.

\item What fraction of counties have a population whose percentage
  under the age of 18 is above 30\%?
\end{enumerate}

\vskip 1em

\item A subdivision of 24 houses has a mean price of \$500,000, a
  median of \$440,000, and a standard deviation of \$30,000. A new
  house is then built in the subdivision that has a price of
  \$700,000.
  \begin{enumerate}
  \item What is the new mean house price?
  \item What is the new standard deviation?
  \item Does the median increase, decrease, or stay the same after the new house is built? Or can no conclusion be made? Explain.
  \end{enumerate}

\vskip 1em

\item Consider a data sample $x_1, \, x_2, \, \ldots, \, x_n$. Let $\bar x$ and $s_x^2$ denote its sample mean and sample variance.
	
		\begin{enumerate}
			\item Suppose that you modify these data by adding a constant $c$ to each observation in the sample, and then multiplying by another constant $ k $, to obtain a modified sample $y_1, \, \ldots, \, y_n$. (That is, $y_i = (x_i + c) \times k$ for each $i$.)
		
			\vskip 0.5em
		
			What are $\bar y$ and $s_y^2$, the sample mean and variance of the modified data? Justify your answer mathematically, using the definition of the sample mean and variance. (We saw a similar question in lecture; here you need to write down the proof yourself.)
		
			\item Finally, suppose we built another modified data sample $z_1, \, \ldots, \, z_n$, where
			$$
				z_i = \frac{x_i - \bar x}{s_x},
			$$
			where $s_x$ is the sample standard deviation of the $x$--data. This procedure
            \textit{standardizes} the original data. What are $\bar z$ and $s_z^2$?	Justify your answer.
\end{enumerate}

\end{enumerate}

\end{document}
