\documentclass{article}    % Specifies the document style.
\renewcommand{\baselinestretch}{1}      % Spacing
\setlength{\oddsidemargin}{0in} \setlength{\evensidemargin}{0in} \setlength{\textwidth}{6.5in}
\setlength{\topmargin}{-.2in} \setlength{\headsep}{0in} \setlength{\textheight}{9.5in}
\pagestyle{empty}
%
\usepackage{color}
\usepackage{amsmath,amsthm}
\usepackage{graphicx}
\usepackage{enumitem}
\usepackage{listings,url}
\lstset{ %
	language=R,                % choose the language of the code
	basicstyle=\footnotesize,       % the size of the fonts that are used for the code
	numbers=left,                   % where to put the line-numbers
	numberstyle=\footnotesize,      % the size of the fonts that are used for the line-numbers
	stepnumber=1,                   % the step between two line-numbers. If it is 1 each line will be numbered
	numbersep=5pt,                  % how far the line-numbers are from the code
	backgroundcolor=\color{white},  % choose the background color. You must add \usepackage{color}
	showspaces=false,               % show spaces adding particular underscores
	showstringspaces=false,         % underline spaces within strings
	showtabs=false,                 % show tabs within strings adding particular underscores
	frame=single,           % adds a frame around the code
	tabsize=2,          % sets default tabsize to 2 spaces
	captionpos=b,           % sets the caption-position to bottom
	breaklines=true,        % sets automatic line breaking
	breakatwhitespace=false,    % sets if automatic breaks should only happen at whitespace
	escapeinside={\%*}{*)},          % if you want to add a comment within
	% your code
	morekeywords={size_t, std, vector ,cout,endl},
	sensitive=false
}

\begin{document}           % End of preamble and beginning of text.
	\begin{center}
		
		\textbf{\large{ENGRD 2700: Basic Engineering Probability and Statistics}\medskip \\ \large{Fall 2019}}
		
		\vskip 0.75em
		
		\textbf{\large{Homework 9}}
	\end{center}
	
	
	\vskip 1em
	
	\noindent Due \textbf{Friday December 13th} at 11:59 pm. Submit to Gradescope by clicking the name of the assignment. See \url{https://people.orie.cornell.edu/yudong.chen/engrd2700_2019fa.html#homework} for detailed submission instructions. 
	
	\vskip 1em
	
	
	The same stipulations from Homework 1 (e.g., independent work, computer code, etc.) still apply. \bigskip
	
	\vskip 1em
	
	\begin{enumerate}[leftmargin=0.55cm]
		
		
		
		\item We want to fit a regression line to the data pairs $(1, 4)$, $(2, 3)$, $(3, 5)$, $(4, 10)$. 
		\begin{enumerate}
			\item Find the line $y = \beta_0 + \beta_1 x$ that minimizes the sum of squared errors.
			\item Compute the $R^2$--value associated with the line you found in part (a).
			\item Construct a $95\%$ confidence interval for the slope coefficient $\beta_1$.
			\item If we perform the hypothesis test
			$$
			H_0: \beta_1 = 0 \qquad \qquad H_1: \beta_1 \neq 0,
			$$
			at the $\alpha = 0.05$ significance level, do we reject $H_0$? Why or why not?\\ \vskip.25em
		\end{enumerate}
		
		\item  Consider once again the dataset \texttt{Quartet.csv} from Homework 1. Import this file into R or RStudio, or similar software.
		\begin{enumerate}
			\item Apply linear regression to each of the four datasets in the file. Attach your code, and report the four lines you obtain. (If you use R, you can use the command \texttt{reg = lm(y $ \sim $ x)} to fit a line between the predictor $ x $ and the response $ y $, and the command \texttt{summary(reg)} to read the results.)
			\item What are the $R^2$--values associated with the four models from part (a)?  Attach your code for computing the $ R^2 $ values. (If you use R, you can find the $ R^2 $ value by reading the ``Multiple R-Squared'' from the output of the command \texttt{summary(reg)}.)
		\end{enumerate}
		

	\end{enumerate}
\end{document}