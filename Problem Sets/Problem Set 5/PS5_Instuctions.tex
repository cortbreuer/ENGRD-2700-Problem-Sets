\documentclass{article}    % Specifies the document style.
\renewcommand{\baselinestretch}{1}      % Spacing
\setlength{\oddsidemargin}{0in} \setlength{\evensidemargin}{0in} \setlength{\textwidth}{6.5in}
\setlength{\topmargin}{-.2in} \setlength{\headsep}{0in} \setlength{\textheight}{9.5in}
\pagestyle{empty}
%
\usepackage{color,url}
\usepackage{placeins}
\usepackage{amsmath,amsthm}
\usepackage{graphicx}
\usepackage{enumitem}
\usepackage{relsize}
\usepackage{float}

\begin{document}           % End of preamble and beginning of text.
	\begin{center}
		
		\textbf{\large{ENGRD 2700: Basic Engineering Probability and
				Statistics\medskip \\ Fall 2019}}\bigskip
		
		\vskip 0.25em
		
		\textbf{\large{Homework 5}} 
	\end{center}
	
	\vskip 1em
	
	\noindent Due \textbf{Friday November 1} at 11:59pm. Submit to Gradescope by clicking the name of the assignment. See \url{https://people.orie.cornell.edu/yudong.chen/engrd2700_2019fa.html#homework} for detailed submission instructions. 
	
	\vskip 1em
	
	
	The same stipulations from Homework 1 (e.g., independent work, computer code, etc.) still apply. \bigskip
	
	\vskip 1em

	\begin{enumerate}
		\item  The joint density function of two random variables is given by
		$$
		f_{X,Y}(x,y) = 
		\begin{cases}
		k(2y+xy) & \text{if } 0 \le x \le 1\text{ and } x \le y \le 1 \\
		0 & \text{otherwise.}
		\end{cases}
		$$
		\begin{enumerate}
			\item What is the constant $k$?
			\item What is the marginal pdf of $X$?
			\item What is the conditional density of $X$, given that $Y = 1/4$?
		\end{enumerate}
		
		\item  The lifetimes $X$ and $Y$ of two batteries (in years) are distributed according to the following joint PDF:
		$$
		f_{X,Y}(x,y) = 
		\begin{cases} 
		8e^{-4x}e^{-2y} & \text{if }x\geq 0 \text{ and } \, y \geq 0 \\ 
		0 & \text{otherwise}
		\end{cases}
		$$
		\begin{enumerate}
			\item Compute the probability that both batteries last at least 2 years.
			
			\item  Find the marginal PDFs $f_X(x)$ and $f_Y(y)$.
			
			\item Compute $E[XY]$.
			
			\item What is $P(X \le \frac{Y}{2})$, the probability that the first battery lasts at most half as long as the second?
			
			\item Are $X$ and $Y$ independent? Why or why not?
		\end{enumerate}
		
		\item For two discrete random variables  $X$ and $Y$, recall the definition of $ f_{X|Y}(x|y) $, the \textit{conditional PMF} of $X$ given that $Y=y$.
		Also recall that conditional PMFs are still PMFs, so they must sum to one; that is,
		$
		\sum_{x} f_{X|Y}(x|y) = 1.
		$
		
		A bakery sells two types of cupcakes:
		red velvet and salted caramel. Let $R$ and $S$ denote the
		number of each type of cupcake an individual customer
		buys. Suppose that $R$ and $S$ are distributed according to
		the following joint PMF:
		
		\begin{center}
			
			\vskip -1em
			
			\begin{tabular}{ r | c | c c c }
				& &  & $s$  & \\
				\hline
				$f_{R,S}(r,s)$  & & 0 & 1   & 2\\
				\hline
				& 0 &  0   & 0.25 & 0.15\\
				$r$ & 1 & 0.20 & 0.10 & 0.10\\
				& 2 & 0.10 & 0.05 & 0.05
			\end{tabular}
		\end{center}

		\begin{enumerate}
			\item  			
			Compute $f_{R|S}(1|1)$, the conditional probability that a customer buys 1 red velvet cupcake, given that (s)he buys 1 salted caramel cupcake. How does this probability compare to $f_R(1)$? 
			
			
			\item Find $f_{R|S}(r|2)$, the conditional PMF of $R$, given that the customer buys $ 2 $ salted caramel cupcakes. (This involves computing $f_{R|S}(0|2)$, $f_{R|S}(1|2)$, and $f_{R|S}(2|2)$.)
			
			
			\item  The \textit{conditional expectation} of $X$, given that $Y = y$, is defined as
			$$
			E[X \,|\, Y = y] = \sum_x x\,f_{X|Y}(x|y).
			$$
			Find $E[R\,|\,S=2]$, the expected number of red velvet cupcakes a customer buys, given that (s)he buys $ 2 $ salted caramel cupcakes.
			
		\end{enumerate}
		
		\item  Given two continuous random variables $X$ and $Y$, recall the definition of $ f_{X|Y} (x|y) $, the \textit{conditional PDF} of $X$ given that $Y = y$. Conditional PDFs are still PDFs, so they must \textit{integrate} to one:
		$
		\int_{-\infty}^{\infty} f_{X|Y}(x|y) dx = 1.
		$ \smallskip
		
		Suppose $X$ and $Y$ are distributed as in the following joint PDF:
		$$
		f_{X, Y}(x, \,y) = \begin{cases}2 ye^{-y(2+x)} & x, \, y \geq 0 \\ 0 & \text{otherwise} \end{cases}
		$$
		For $y > 0$, compute $f_{X|Y}(x|y)$. 
		
		\item The file \texttt{Bwages.csv} contains hourly wages of 1473 randomly selected individuals living in Belgium in 1994. For the questions that follow, attach your plots, as well as any code used to generate them.
		
		\begin{enumerate}
			\item Import this dataset into RStudio, and generate a histogram of the data. 
			
			\item  Construct a normal Q--Q plot, by hypothesizing
			that the data originate from a Normal($\bar x$, $s^2)$ distribution,
			where $\bar x$ and $s^2$ are the sample mean and sample variance,
			respectively. Overlay the line $y=x$ onto your plot.  (See
			Recitation 6 for an example of how to do this.) 		
			
			\item Does the normal distribution appear to be a reasonable fit for the data? Why or why not? 
			

			\item  We'll now attempt to fit a \textit{lognormal distribution} to the data. If $X \sim \text{Lognormal}(\mu, \, \sigma^2)$, it has PDF
			$$
			f(x) =\frac{1}{x\sigma\sqrt{2\pi}}e^{-(\ln(x) - \mu)^2/(2\sigma^2)} \qquad x > 0
			$$
			The lognormal random variable gets its name because $\ln(X)$ is normally distributed (with mean $\mu$ and variance $\sigma^2)$. Construct a lognormal Q--Q plot, by hypothesizing that the data originate from the Lognormal$(2.31, \, 0.41)$ distribution. Use R's \texttt{qlnorm} function to compute theoretical quantiles, and set \texttt{meanlog} to 2.31 and \texttt{sdlog} to 0.41. Overlay the line $y=x$ onto your plot.
			
			\item Does the lognormal distribution appear to be a better fit for the data? Where could the fit be improved? Comment on what you see. 
			
		\end{enumerate}		
	\end{enumerate}
	
\end{document}
