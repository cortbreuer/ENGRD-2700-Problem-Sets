\documentclass[letterpaper]{article}

\usepackage[ddmmyy]{datetime}
\usepackage[margin=1.25in]{geometry}
\usepackage{fancyhdr}
\usepackage{amsmath}

\pagestyle{fancy}
\fancyhf{}
\rhead{Cort Breuer}
\chead{\today}
\lhead{ENGRD 2700}
\rfoot{\thepage}

\begin{document}

\vspace*{6pt}

\noindent \textbf{\huge{Problem Set 3}}

\bigskip

\section*{Question 1}

Lifetime of a light bulb $T$ is given by $P(T>t)=e^{-t/3}$ for all $t \geq 0$. The bulb has lasted $x$ years, so the conditional probability that it will last at most $x + 2$ years is given by the conditional probability equation.

 $$P(A|B) = \frac{P(A \cap B)}{P(B)}$$

$$P(A \cap B) = \int_x^{x+2} e^{-t/3} dt = -\frac{1}{3} e^{-t/3} \Big|_x^{x+2} = \frac{1}{3} e^{-x/3} - \frac{1}{3} e^{-(x+2)/3}$$

$$P(B) = \int_0^{x} e^{-t/3} dt = -\frac{1}{3} e^{-t/3} \Big|_0^{x} = - \frac{1}{3} e^{-x/3} + \frac{1}{3} e^{-0/3} = \frac{1}{3} - \frac{1}{3} e^{-x/3}$$

$$P(A|B) = \frac{\frac{1}{3} e^{-x/3} - \frac{1}{3} e^{-(x+2)/3}}{\frac{1}{3} - \frac{1}{3} e^{-x/3}} = \frac{e^{-x/3} - e^{-x/3} e^{-2/3}}{1 - e^{-x/3}}$$






\newpage

\section*{Question 2}

\subsection*{Part A}

The probability that both children are female, assuming the first is female, is given by the conditional probability equation $P(A|B)$.

$$P(A|B) = \frac{P(A \cap B)}{P(B)} = \frac{.25}{.5} = .5$$

\noindent There is a probability of .5 that given the first child is female that the second will also be female.

\subsection*{Part B}

Here the conditional probability equation is used again, but the order of  children no longer matters.

$$P(A|B) = \frac{P(A \cap B)}{P(B)} = \frac{.25}{.75} = .333$$

\noindent There is a probability of .333 that given as least one child is female that the other will also be female.

\subsection*{Part C}

$$P(\text{ff } | \geq \text{1 Katie}) = \frac{P(\text{ff } \cap \geq \text{1 Katie})}{P(\geq  \text{1 Katie})} = \frac{.25}{.75}$$

\newpage

\section*{Question 3}

\subsection*{Part A}

A = {bag is identified as dangerous}

\noindent B = {bag contains explosives}

$$P(A|B) = \frac{P(A \cap B)}{P(B)} = \frac{}{10/4000000}$$

\subsection*{Part B}

\subsection*{Part C}

\newpage

\section*{Question 4}

\subsection*{Part A}

The set $\mathcal{X}$ of possible values for $X$ is given by $S = \{ 1, 2, 4, 8, 16, 32, ...\}$ (or by $2^x$ where x can take all integers starting at 0).

\subsection*{Part B}

The PMF of $X$ is given by $p_{X}(x) = P(X = x)$.



\subsection*{Part C}

\subsection*{Part D}

\subsection*{Part E}

\newpage

\section*{Question 5}

\subsection*{Part A}

$$

f(x) = \begin{cases}
    P(X = -2) & = .2  \\
    P(X = -1) & = .1  \\
    P(X = 0) & = .4  \\
    P(X = 1) & = .1  \\
    P(X = 2) & = .2
\end{cases} $$

\subsection*{Part B}

$$E[X] = \sum_{x \in \mathcal{X}} x \cdot P_X (X = x) = (-2 \cdot .2) + (-1 \cdot .1) + (0 \cdot .4) + (1 \cdot .1) + (2 \cdot .2) = -.4 - .1 + 0 + .1 + .4 = 0$$

\subsection*{Part C}

\newpage

\section*{Question 6}

\subsection*{Part A}

\subsection*{Part B}

\subsection*{Part C}

\subsection*{Part D}

\subsection*{Part E}

\end{document}
