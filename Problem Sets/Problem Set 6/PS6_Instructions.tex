\documentclass{article}    % Specifies the document style.
\renewcommand{\baselinestretch}{1}      % Spacing
\setlength{\oddsidemargin}{0in} \setlength{\evensidemargin}{0in} \setlength{\textwidth}{6.5in}
\setlength{\topmargin}{-.2in} \setlength{\headsep}{0in} \setlength{\textheight}{9.5in}
\pagestyle{empty}
%
\usepackage{color,url}
\usepackage{amsmath,amsthm}
\usepackage{graphicx}
\usepackage{enumitem}
\usepackage{listings}

\newcommand{\answer}[1]{{\color{blue}#1}}
\newcommand{\grade}[1]{\textcolor{red}{#1}}
%\renewcommand{\grade}[1]{ }	%to remove grading scheme, just make this link active
%\renewcommand{\answer}[1]{ } %%to remove answer , just make this link active


\definecolor{darkgreen}{rgb}{0,.4,0}
\newcommand{\yccomment}[1]{\textbf{\color{darkgreen} Yudong --- #1}}
%\renewcommand{\yccomment}[1]{ } %%to remove Yudong'scomment , just make this active



\begin{document}           % End of preamble and beginning of text.
	\begin{center}
		
		\textbf{\large{ENGRD 2700: Basic Engineering Probability and Statistics}\medskip \\ \large{Fall 2019}}
		
		\vskip 0.75em
		
		\textbf{\large{Homework 6}}
	\end{center}
	
	\vskip 1em
	
	\noindent Due \textbf{Friday Nov 15} at 11:59 pm. Submit to Gradescope by clicking the name of the assignment. See \url{https://people.orie.cornell.edu/yudong.chen/engrd2700_2019fa.html#homework} for detailed submission instructions. 
	
	\vskip 1em
	
	
	The same stipulations from Homework 1 (e.g., independent work, computer code, etc.) still apply. \bigskip
	
	\vskip 1em
	
	\begin{enumerate}[leftmargin=0.55cm]
		
			\item Compute the covariance and correlation of $X$ and $Y$, when these continuous random variables have joint density function given by
		$$f_{X, Y}(x, y) = \begin{cases}
		1/8 & x\in [0, 1), y \in [0, 1) \\
		1/8 & x\in [1, 2), y \in [1, 2) \\
		3/8 & x \in [0, 1), y \in [1, 2), \text{ or } x \in [1, 2), y \in [0, 1) \\
		0 & \text{otherwise.}
		\end{cases}
		$$
		
		
		
		\item Suppose there are two stocks $X$ and
		$Y$. The annual returns of the stocks are normally distributed, with
		the same mean $\mu_1=\mu_2=12$ and variances $\sigma_1^2=9$ and
		$\sigma_2^2=16$, respectively. 
		\begin{enumerate}
			
			\item Assume that $ X $ and $ Y $ are independent. Suppose you hold one share of $X$ and one share of $Y$. Compute the probability that your total annual return is greater than
			25.
			
			\item Now assume that $ X $ and $ Y $ are negatively correlated
			with correlation $\rho=-0.6$. It turns out that the sum of two (possibly
			correlated) normal random variables is still normallly
			distributed. Compute the probability in (a). 
			
			
			\item Still assume that $ \rho = -0.6 $. Suppose you can only purchase one share of $X$
			and $Y$ in total. That is, your returns from the two stocks are $w X$ and $(1-w)Y$,
			where $0 \leq w \leq 1$. What choice of $w$ minimizes the variance of
			your total return?
			
			\item Now assume that $X$ and $Y$ are perfectly negatively
			correlated (i.e., $\rho = -1$), what value of $w$ minimizes the
			variance of your total return? 
			
			
			\item Would you prefer to invest under the conditions
			of part (c) or part (d)? Why? 
			
		\end{enumerate}
		\item When we have sample data we can compute the {\em sample} covariance and the  {\em sample} correlation. In particular, suppose we
		have data pairs $((X_i, Y_i), i = 1, 2, \ldots, n)$. The sample
		covariance is defined as
		$$q_{X,Y} = \frac1 {n-1} \sum_{i=1}^n [(X_i - \bar X) (Y_i - \bar
		Y)],$$
		and the sample correlation is defined as
		$$r = \frac{q_{X, Y}}{s_X s_Y},$$
		where $s_X^2$ and $s_Y^2$ are the sample variances of the $X_i$'s and
		$Y_i$'s, respectively. 
		
		\begin{enumerate}
			\item Consider the dataset \texttt{DataForSunglasses.csv}. Generate a scatter plot for ice cream sales vs temperature, and compute their sample correlation. Do the same thing for sunglasses sales vs temperature, and for sunglasses sales vs ice cream sales. (That is, you need to generate three plots and compute three correlations.)
			
			\item You should see a positive correlation between
			sales of sunglasses and ice cream sales. Does this mean that ice
			cream makes people sensitive to sunlight? Explain in one sentence.
		\end{enumerate}
		
		
		\item Suppose you flip a fair coin 100 times
		
		\begin{enumerate}
			\item Use the Central Limit Theorem to approximate the probability that heads appears at most 46 times. 
			
			
			\item Write down an expression for the
			exact probability in part (a), and compute it. (In
			R, you can use \texttt{pbinom}.)
			
			
			\item Our approximation from part (a) is not
			very accurate.  This is primarily because we are using a
			continuous random variable (the normal) to approximate a
			discrete random variable (the binomial). To improve our approximation, we can use the fact that$$
			P(S_{100}\leq 46)=P(S_{100}\leq 46+c)
			$$
			for any constant $0\leq c <1$, since $ S_{100} $ can only take integer values. It turns out that $c= 0.5$
			works very well in practice, and so to approximate the
			probability of seeing at most 46 heads, we can apply the
			Central Limit Theorem to $P(S_{100}\leq 46.5)$ instead.  Do
			this, and compare the approximation you obtain here to the one
			in part (a). The procedure is referred to as a \emph{continuity correction}.
			
			
			
		\end{enumerate}
		
		\item An insurance company looks at
		the records for millions of homeowners and conclude the
		probability of fire in a year is  $0.01$ for each house and
		the loss should a fire occur is \$10,000. Thus the expected loss from
		fire for each house is \$100. Assume the fires are
		independent. The company plans to sell fire insurance for
		\$120 (which is the expected loss plus \$20). If a house owner purchases the insurance and his/her house catches fire, the company will cover for the loss.
		\begin{enumerate}
			\item If the company sells the insurance policy to 10 houses, what is the expected total profit of the company?
			
			
			\item Compute the probability of bankruptcy for the company, that is,
			when total profit is negative.
			
			
			\item Use the Central Limit Theorem to approximately compute the probability of bankruptcy if the company sells the policy to 1 million houses.
			
		\end{enumerate}
		
		
		\item A light bulb has a lifetime that is exponentially distributed with rate parameter $\lambda = 5$. Let $L$ be a random variable denoting the sum of the lifetimes of 50 such bulbs. Assume that the bulbs are independent. 
		\begin{enumerate}
			\item Compute $E[L]$ and Var$(L)$.
			
			\item Use the Central Limit Theorem to approximate $P(8 \leq L \leq 12)$.
			
			\item Use the Central Limit Theorem to find an interval $(a, b)$, \emph{centered at $E[L]$}, such that
			$$
			P(a \leq L \leq b) = 0.95.
			$$
			That is, your interval should be of the form $(E[L] - c, E[L] + c)$, for some constant $c > 0$. 
		\end{enumerate}
		
		
	\end{enumerate}
\end{document}